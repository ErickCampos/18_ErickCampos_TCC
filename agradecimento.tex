Primeiramente, agradeço à minha família, por toda ajuda e apoio que foram de
fundamental importância tanto para a minha formação pessoal quanto profissional.
Em especial, à minha mãe Cláudia Modesto e meu padrasto Edevaldo Trindade, por
sempre confiarem e me apoiarem em todas as minhas decisões. Agradeço também aos
meus avós e a todos os meus tios e primos, que sempre estiveram, direta ou
indiretamente, ao meu lado. Amo vocês. Um dia espero retribuir tudo o que vocês
já fizeram por mim.

Agradeço em especial à Suzane Santos, pelo carinho, companhia, ajuda e
conselhos. Sem dúvidas, você foi uma das pessoas mais importantes para mim nos
últimos quatro anos. Não se esqueça que apenas vou atravessar a ponte do básico 
e que sempre vou estar ao seu lado. Amo você, mulher.  

Aos meus orientadores Cassio Batista e Nelson Neto, pela confiança no trabalho
que venho desenvolvendo há dois anos; pela paciência e também pelos ``puxões de
orelha'' que me ajudaram a crescer academicamente. 

Agradeço também aos meus amigos Daniel Breno, Israel Lucas, Rodrigo Lima pelas
inúmeras ajudas ao longo do curso. Certamente não poderia deixar de agradecer à
todos os demais colegas do curso de Engenharia da Computação e Engenharia de
Telecomunicações, os quais não conseguiria listar e, por isso, não ouso citar
nomes.

Aos professores da Engcomp/FCT, sem os quais não teria a disciplina e o
conhecimento necessários para o desenvolvimento deste trabalho.

Por fim, Agradeço aos voluntários que aceitaram participar dos testes, sem as quais
os resultados e conclusões deste trabalho não teriam sido possíveis.

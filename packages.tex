%% Packages used for the TCC

%\usepackage{booktabs}

%%%% Definitions and New commnads %%%%

\newcommand{\sen}{\operatorname{sen}}
\newcommand{\mbeq}{\overset{!}{=}}

\renewcommand\Re{\operatorname{Re}}
\renewcommand\Im{\operatorname{Im}}

%%%% New control sequences %%%%

\newcommand{\redeq}[1] {\textcolor{red}{#1}}
\newcommand{\blueeq}[1] {\textcolor{blue}{#1}}
\newcommand{\att}[1] {\textcolor{red}{#1}}
% Used Packages:
\usepackage{tikz}
\usepackage{float}
\usepackage{subfig} % casso
\usepackage{steinmetz}
\usetikzlibrary{arrows,shapes,shapes.multipart}
\usepackage[brazil]{babel}
\usepackage[T1]{fontenc}
\usepackage[utf8]{inputenc}
\usepackage{amsmath}
\usepackage{amsfonts}
%\usepackage{enumitem} % To adjust lists
\usepackage{verbatim} % Multi-line comments
\usepackage{mathabx} % Package that contains the circular convolution symbol
\usepackage{graphicx}
\usepackage{caption}
%\usepackage{subcaption} % casso
\usepackage{units}
\usepackage{adjustbox}
\usepackage{dirtytalk}
\usepackage{csquotes}
\usepackage{enumerate}
\usepackage{url} 
\usepackage[pdfusetitle]{hyperref}
\usepackage{breakurl}
\usepackage{array}  
\usepackage{eurosym}
\usepackage{tabularx}

\usepackage{siunitx} %símbolo OHM

% \begin{casso}
% https://tex.stackexchange.com/questions/299969/titlesec-loss-of-section-numbering-with-the-new-update-2016-03-15
\usepackage{titlesec}
\usepackage{etoolbox}

\makeatletter
\patchcmd{\ttlh@hang}{\parindent\z@}{\parindent\z@\leavevmode}{}{}
\patchcmd{\ttlh@hang}{\noindent}{}{}{}
\makeatother
% \end{casso}

%\usepackage[backend=bibtex8]{biblatex}
%\makeatletter
%\providecommand\@enumctr{}
%\makeatother

\ifpdf

\ifdefined\hyperref
\else
\usepackage[pdftex,colorlinks]{hyperref}
\fi

\hypersetup{%
pdftitle={Some title},
pdfauthor={Your name - LaPS - UFPA},
pdfkeywords={DSP,Signal},
pdfstartview={FitH}, %% <--
urlcolor=black,
%linkcolor=blue,
linkcolor=black,
%citecolor=red,
citecolor=black,
}
\newcommand*{\captionsource}[2]{%
  \centering{
  \caption[{#1}]{%
    #1%
   % \\\hspace{\linewidth}%
   \\ 
	\textbf{Fonte:} #2%
  }}
}


% Ensiar o Latex a separar silabas
\hyphenation{DMT En-ge-nhei-ro}

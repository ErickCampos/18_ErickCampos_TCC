\begin{chapter}{Referencial Teórico}
Este capítulo tem como objetivo apresentar o referêncial teórico das ferramentas
mais importantes para o desenvolvimento do dispositivo proposto neste trabalho,
como i) o efeito piezoelétrico; ii) Conectores de áudio; iii) captura de sinais
através da interface de áudio do computador; e, finalmente iv) ferramenta para a
emulação dos eventos de clique do \textit{mouse}

\begin{section}{Conectores de Áudio}

Sinais de áudio podem ser transmitidos através de vários tipos de conectores.
Por exemplo, é possivel reproduzir os sinais de áudio produzidos por uma
guitarra para uma caixa acústica amplificada. Geralmente utiliza-se o conector
P10 para realizar a transmissão da guitarra para a caixa acústica. Já para fones
de ouvidos, por exemplo, é utilizado o conector P2. Segundo uma matéria
disponível na BBC News~\cite{BBC}, o conector P2 é utilizado desde o século XIX.
Os conectores mais utilizados são os conectores TS e TRS que serão brevemente
descritos a seguir.

\begin{subsection}{Conector TS}

ok

\end{subsection}

\begin{subsection}{Conector TRS}
fala meu parceiro
\end{subsection}

\end{section}
\end{chapter}

\begin{chapter}{Protótipo}
O acionador baseado em sopro foi desenvolvido com a intenção de ser
\textit{open-source} e de baixo custo, para que mais pessoas possam ter acesso
a essa ferramenta que pode ser utilizada como método alternativo para o clique
do \textit{mouse}. O circuito do protótipo desenvolvido neste trabalho é mostrado 
na Figura~\ref{fig:circuito}

\begin{figure}[!h]
	\centering
	\begin{minipage}[c]{\textwidth}
	\centering
	\includegraphics[width=0.9\linewidth]{fig/acionador}
	\caption{Esquemático do acionador baseado em sopro.}
	\label{fig:circuito}
	\end{minipage}
\end{figure} 

O princípio de funcionamento do acionador externo proposto é relativamente
simples. A percepção do sopro é realizada graças ao piezo encapsulado em um
disco redondo que fica acoplado na parte superior do dispositivo. Quando o
usuário realiza o sopro diretamente no piezo, há um estresse mecânico que produz
uma diferença de potencial nas extremidades do material piezoelétrico. Através
da percepção dessa tensão produzida no piezo é possível gerar um evento de
clique de \textit{mouse} no computador. O resistor colocado em paralelo com o
piezo é responsável por determinar a sensibilidade com que o transdutor gera
tensão após ser submetido a algum tipo de estresse mecânico. Portanto quanto
maior o valor de resistência do resistor, maio será a sensibilidade do piezo.
No protótipo desenvolvido utilizou-se um resistor de 10~M\si{\ohm} o que
proporcionou uma boa sensibilidade no piezo. Foram testados resistores com
valores de resistência maior que 10~M\si{\ohm}, como 15~M\si{\ohm}, 20~M\si{\ohm} e
30~M\si{\ohm}, contudo a sensibilidade com resistência maiores que 15~M\si{\ohm}
deixava o transdutor muito suscetível a detectar estresses mecânicos com
intensidades não desejadas. Já resistências menores que 15~M\si{\ohm} não
alterava significativamente a sensibilidade do piezo, quando comparado com o
resistor de 10~M\si{\ohm}. 

O sinal da tensão gerado nas extremidades do material piezoelétrico é analógico.
Contudo, como o clique do \textit{mouse} possui uma natureza binária, pois há
apenas dois estados possíveis --- clique ativado e clique não ativado ---, não
seria possível utilizar diretamente essa tensão produzida como forma de ativação
do clique devido a grande quantidade de ruído que esse sinal possui. Por isso 
houve a necessidade de digitalizar a saída do acionador.

A solução encontrada para realizar a tarefa de ``converter'' o sinal analógico
do acionador para um sinal digital, foi a utilização de um amplificador
operacional LM358. Esse componente eletrônico possui a função de comparar duas
tensões que determinam a saída do acionador. Há uma tensão de referência
, que pode ser variável graças a um potenciômetro de 10~k\si{\ohm}, e a tensão
produzida pelo transdutor piezo. O amplificador operacional atua, neste caso,
como um comparador de tensões. Quando a tensão de referência é superior a tensão
do piezo, a saída do amplificador é de nível lógico baixo. Contudo, quando a
tensão de referência é inferior a tensão do piezo, a saída do amplificador é de
nível lógico alto. Não há possibilidade de a saída do amplificador ser diferente
desses dois casos. Dessa forma, a leitura do sinal de tensão produzido pelo sopro
, com o auxílio do piezo, é binária, o que ajuda a implementar a emulação dos
eventos de clique. É importante ressaltar que como o valor de referência 
pode ser ajustável graças ao potenciômetro, esse componente também atua como um
ferramenta de ajuste de sensibilidade. Sendo assim, a intensidade do sopro
necessária para que a saída do amplificador seja de nível lógico alto,
configurando assim um clique, pode ser determinada pela tensão de referência
ajustada pelo potenciômetro.

Como um dos objetivos do acionador é ser de baixo custo, a cominicação entre o
acionador desenvolvido e a computador é realizada através da interface de áudio
P2 Jack. Para gerar um pulso na interface de áudio, basta curto circuitar o pino
de GND com o \textit{Right} e \textit{Left} do P2 Jack do acionador. Para fazer
isso foi necessário a utilização de uma porta lógica inversora do componente
SN7404. Essa porta lógica captura o sinal da saída do LM358 somente quando a
saída do amplificador é de nível lógico alto, pois a saída desse componente
também é utilizado como alimentação do SN7404. Portanto, quando a saída do
amplificador é de nível lógico baixo, o inversor não funciona, pois não há
tensão suficiente para alimentar esse componente. Quando a saída do LM358 é de
nível lógico alto, o inversor é alimentado com uma tensão suficiente para
realizar p funcionamento correto. Essa tensão é utilizada como entrada do
SN7404 que inverte o sinal de entrada para nível lógico baixo. A saída do
inveror é concetada nos pinos  \textit{Right} e \textit{Left} do P2 Jack --- que
estão curto circuitados --- produzindo gerando um pulso na entrada da interface
de áudio do computador. A partir desse pulso reconhecido pelo computador é
possível realizar a implementação da emulação dos evento de clique do
\textit{mouse}. É importante ressaltar que a tensão de sáida do amplificador
operacional, quando está em nível lógico alto, é bem próximo de 9~V. Essa tensão
é superior a tensão recomenda (5.5~V) pela fabicante do componente~\cite{7404}. 
Para diminuir essa tensão para a recomendada, foi utilizado um LED em série com
um resistor de 300~\si{\ohm}. Quando a saída do amplificador é de nível lógico
alto, há uma queda de tensão do catodo do LED que é suficiente para alimentar o
inversor.Sendo assim além de ser utilizado como divisor de tensão, o LED também
funciona como um \textit{feedback} visual, indicando quando o acionador foi
ativado pelo sopro do usuário.  


\begin{section}{Confecção da Placa de Circuito Impresso}
Para a construção da placa do acionador baseado e sopro utilizou-se o
\textit{software} KiCad~\cite{kicad}, uma ferramenta de código aberto,
multiplataforma e bastante utilizada por desenvolvedores. Todo o projeto da
construção do acionador proposto está disponível em~\cite{ErickGit} para que
qualquer pessoa possa acessar o projeto e, se desejar, construir seu próprio
acionador. É importante ressaltar que alguns componentes, como o piezo, não
possuíam biblioteca nativa no Kicad. Por isso foi necessário desenvolver uma 
biblioteca para cada um desses componentes.

Após a construção do \textit{layout} do circuito no Kicad é que se iniciou a
etapa de confecção da placa. Foi utilizada a técnica de transferência térmica,
um método bem primitivo bastante suscetível a erros, porém bastante prático e
barato. O circuito é impresso em papel fotográfico em uma impressora a laser na
melhor qualidade possível. Em seguida o circuito é transferido para uma placa de
fenolite ou de fibra de vidro através de um processo térmico, onde, a placa é
pressionada por uma superfície em temperatura relativamente alta. No caso do
Protótipo construído neste trabalho, foi utilizado um ferro de passar roupa para
realizar esse processo térmico. É importante pressionar bastante o ferro de
passar roupas contra a placa e realizar movimentos em todas as direções a fim 
de transferir todo o \textit{layout} do circuito para a placa. Após o processo
térmico é necessário retirar da placa o papel fotográfico que continha o 
circuito. Se algum trecho do circuito não for transferido para a placa --- isso
é muito comum de acontecer --- é possível corrigir com um pincel
marcador permanente de CD/DVD as trilhas não transferidas do circuito.  

O processo seguinte a transferência do circuito para a placa é a corrosão da
placa. Para isso é utilizada uma solução de percloreto de ferro. Essa solução é
muito fácil de ser encontrada, pois muitas lojas de eletrônica vendem esse
produto. Geralmente o percloreto de ferro é vendido em pó, porém é necessário
apenas misturar o produto com água para obter a solução desejada. O placa com o
circuito recém transferido é mergulhada nessa solução e toda a placa é corroída
restando apenas o \textit{layout} do circuito desejado. Apos a corrosão é
realizado a limpeza do circuito com uma água corrente e detergente a fim de
retirar todas as impurezas do circuito. A etapa seguinte consiste em furar a
placa nos locais desejados e realizar a soldagem dos componentes da placa.
Realizado todas as etapas descritas a placa está totalmente finalizada e pronta
para ser utilizada. 

\begin{figure}[!h]
	\centering
	\begin{minipage}[c]{\textwidth}
	\centering
	\includegraphics[width=0.45\linewidth]{fig/puff2}
	\caption{Placa de circuito impresso do acionador baseado em sopro.}
	\label{fig:placa}
	\end{minipage}
\end{figure} 

A Figura~\ref{fig:placa} mostra a placa confeccionada neste trabalho. É possível
notar que o dispositivo é relativamente pequeno. Mede cerca de 7$\times$7
centímetros.       


\end{section}


\begin{section}{Suporte de Sustentação do Acionador}

O maior desafio encontrado no desenvolvimento deste trabalho foi de encontrar
uma forma que o usuário conseguisse realizar o sopro diretamente no piezo, pois 
esse é o componente capaz de perceber o sopro do usuário. Os acionadores
baseados em sopro disponíveis no mercado utilizam soluções com tubos de PVC onde
a pessoa sopra através desse tubo para conseguir utilizar o acionar. Contudo,
esse tipo de solução restringe demais a possibilidade de mais de uma pessoa
utilizar a mesma ferramenta, pois para utilizar esse tipo de ferramenta é
necessário manter contato direto com a boca no tubo e isso não é recomendado
devido as questões de higiene pessoal e prevenção de doenças que podem ser 
transmitidas pela saliva~\cite{Li2000}.

Algumas soluções, como em~\cite{CorpPuff}, utilizam uma espécie de 
\textit{headset} para sustentar os tubos de PVC próximos a boca do usuário. Isso
serviu como base para a construção do suporte de sustentação construído neste
trabalho. Inicialmente a ideia era desenvolver uma ferramenta baseada em
\textit{headsets} como mostrado na Figura~\ref{fig:headset}.

\begin{figure}[!h]
	\centering
	\begin{minipage}[c]{\textwidth}
	\centering
	\includegraphics[width=0.4\linewidth]{fig/heaset}
	\caption{\textit{headsets} que seria utilizado como base.}
	\label{fig:headset}
	\end{minipage}
\end{figure} 

A intenção era desenvolver toda a armação do suporte utilizando arame recozido.
O estrutura ficaria acoplada na parte de trás da cabeça do usuário se estenderia
pela orelha e as duas extremidades da estrutura seria responsável por segurar o
acionador próximo a boca do usuário. Contudo, essa solução poderia ser
bastante incômoda, pois o peso do acionador sustentado pela estrutura em volta
da orelha do usuário que poderia gerar muito desconforto e insatisfação.

A ideia de utilizar uma ferramenta em forma \textit{headset} não foi descartada.
Surgiu a ideia de utilizar como base um \textit{headset} semelhante ao mostrado 
na Figura~\ref{fig:fone}. 
 
\begin{figure}[!h]
	\centering
	\begin{minipage}[c]{\textwidth}
	\centering
	\includegraphics[width=0.4\linewidth]{fig/fone}
	\caption{\textit{headset} utilizado como base para o suporte de sustentação
do acionador.}
	\label{fig:fone}
	\end{minipage}
\end{figure}


Essa foi a ideia utilizada para desenvolver o suporte de sustentação do
acionador. Os arames recozidos pensados na solução anterior foram reaproveitados
para a estrutura desenvolvida. A função dos arames é de manter fixo o acionador
próximo a boca do usuário. Os arames foram acoplados nas laterais do
\textit{headset} utilizado e foram curvados de uma forma que as extremidades
dos arames ficassem na frente da boca do usuário. Duas garras do tipo jacaré
focaram colocadas nas extremidades  dos arames a fim de acoplar o acionador na
estrutura construída. O resultado do suporte desenvolvido é mostrado na
Figura~\ref{fig:suporte}. 

\begin{figure}[!h]
	\centering
	\begin{minipage}[c]{\textwidth}
	\centering
	\includegraphics[width=0.5\linewidth]{fig/erick}
	\caption{Um usuário testanto o acionador baseado em sopro em conjunto com o
suporte de sustentação. Note que o LED de \textit{status} fica no campo de visão
do usuário. Sendo assim, o suporte não compromete a visualização do LED que
informa quando o acionador foi ativado pelo sopro.}
	\label{fig:suporte}
	\end{minipage}
\end{figure}



Com essa estrutura o acionador sempre ficará na frente da boca do usuário,
independente da posição da cabeça da pessoa. É importante também ressaltar que,
como o \textit{headset} utilizado possui uma estrutura de ajuste de tamanho, o
suporte pode ser utilizado por varias pessoas bastando apenas ajustar o tamanho
desejado para cada tamanho de cabeça. 

\end{section}

\begin{section}{Controle do Cursor do \textit{Mouse}}

O dispositivo baseado em sopro proposto neste trabalho foi desenvolvido para ser
utilizado em conjunto com algum \textit{software} que permite controlar o
ponteiro do \textit{mouse} através de métodos não convencionais. Neste trabalho
não foi desenvolvido nenhum programa que realiza essa função de controle do
cursor. No entanto, foi utilizado o \textit{software} eViacam~\cite{eviacam}.
Esse programa, distrubuído livremente, é multiplataforma e permite controlar os
movimentos do cursor do \textit{mouse} através dos movimentos da cabeça
capturados por uma \textit{webcam}.

O eViacam foi escolhido por ser gratuito, prático (pois é necessário apenas de
um computador com uma \textit{webcam} para funcionar) e possuir muitas opções de
personalização das configurações que podem adaptadas para vários perfis de
usuários.  

 


\end{section}

\end{chapter}

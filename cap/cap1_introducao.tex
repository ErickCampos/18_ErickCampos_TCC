\begin{chapter}{Introdução}

\section{Contextualização}
Grande parte das tecnologias disponíveis no mercado já estão economicamente
acessíveis para uma grande parte da população. O uso de dispositivos eletrônicos
--- como os \textit{smatphones} e os computadores pessoais --- para o auxílio de
diversas tarefas tornou-se mais recorrente no cotidiano das pessoas. Com mais
pessoas utilizando essas ferramentas está surgindo inúmeras formas de melhorar a
interação entre usuários e aparelhos eletrônicos.

Sistemas de reconhecimento ativo de gestos(AGR, do inglês \textit{actice gesture
recognition})~\cite{Darrel96}, reconhecimento automático de voz (ASR, do inglês
\textit{automatic speech recognition})~\cite{Taylor09}, síntese de voz (TTS, do inglês \textit{
text-to-speech})~\cite{Huang01}, e acionadores externos são utilizados para melhorar a 
interação humano-computador (IHC). Um sistema AGR é responsável por aplicar
técnicas de computação visual para realizar o processamento \textit{frames} de
vídeos de entrada e definir, então, na saída, qual a ação referente ao movimento
motor realizado por uma determinada parte do corpo do usuário. O ASR é o sistema
que recebe um sinal de fala digitalizado como entrada e gera um texto transcrito
na saída. O sistema TTS possui a função de gerar um sinal de voz sintetizado a
partir de um texto posto como entrada. Já os acionadores externos são
equipamentos que auxiliam as pessoas com deficiência (PCD) a utilizarem
aparelhos eletrônicos. Essas ferramentas ajudam  no controle de dispositivos
eletrônicos promovendo comodidade e praticidade às pessoas e são normalmente
enquadradas no conceito de Tecnologia Assistiva.

A Tecnologia Assistiva (TA) é uma área do conhecimento, de característica
interdisciplinar, que engloba produtos, recursos, metodologias, estratégias,
práticas e serviços que objetivam promover a funcionalidade, relacionada à
atividade e participação, de pessoas com deficiência, incapacidades ou
mobilidade reduzida, visando sua autonomia, independência, qualidade de vida e
inclusão social~\cite{cat09}.

Através da TA é possível reduzir as dificuldades vivenciadas por pessoas que
necessitam de soluções que não as deixem à margem da utilização de aparelhos
eletrônicos. Visando diminuir a exclusão digital imposta às PCD pela dificuldade
ou total incapacidade para manipular certos equipamentos, a acessibilidade é
vista como elemento fundamental para elevar a autoestima e o grau de
independência dessas pessoas.

\section{Justificativa}

Segundo dados da Organização Mundial de Saúde (WHO, do inglês \textit{World
Health Organization}), aproximadamente 15\% da população mundial possui algum
tipo de deficiência~\cite{WHO15}. Esse número é realmente expressivo, pois
revela que, em uma população de 7,6 bilhões de pessoas, cerca de um sétimo
(1~bilhão de pessoas) é portadora de deficiência. A WHO também afirma que, em
2013, 80\% das pessoas com deficiência viviam em países ainda em
desenvolvimento, o que sugere que o predomínio da condição de deficiência está
bastante relacionado com a situação econômica dos países.

No Brasil, segundo o censo realizado em 2010 pelo Instituto Brasileiro de
Geografia e Estatística (IBGE), aproximadamente 23,9\% da população (cerca de
uma entre quatro pessoas, um total de 46~milhões de habitantes) declarou ter
alguma deficiência~\cite{tIBGE}. Os dados também mostram que, desse total, quase
7\% (cerca de de 13,2~milhões) apresentam dificuldades motoras. A
Tabela~\ref{tab:ibge} mostra o perfil da população brasileira com deficiência.

\begin{table}[!h]
\centering
\caption{Perfil da população brasileira com deficiência.}
\label{tab:ibge}
\def\arraystretch{1.25}
\begin{tabular}{lccr}
	\hline
	\hline
	\textbf{Deficiência} & \textbf{Descrição} & \textbf{Número de Pessoas} &
\textbf{Porcentagem} \\
	\hline
	Visual    & Cegueira ou dificuldades gerais   & 35.774.392  & 18,754 \%  \\
	Motora    & Paralisia ou dificuldades gerais  & 13.265.599  & 6,95 \% \\
	Auditiva  & Surdez ou dificuldades gerais     & 9.717.318   &  5,094 \%  \\
	Cognitiva & Problemas mentais ou intelectuais & 2.611.536   &  1,369 \%  \\ 
	\hline
	\hline
\end{tabular}
\end{table}

Apesar de já existir inúmeros instrumentos voltados para Tecnologia Assistiva
como cadeiras de rodas e \textit{softwares} que facilitam a utilização de
computadores, grande parcela das PCD ainda não têm acesso a essas ferramentas. A
Organização Mundial de Saúde estima, por exemplo, que em países
subdesenvolvidos, aproximadamente 15\% das PCD têm acesso a essas Tecnologias
Assistivas. Um fator que pode contribuir para esse cenário são os altos preços
de algumas dessas tecnologias. Os acionadores externos, por exemplo, apesar de
existir uma grande variedade de tipos e funcionalidades, possuem um preço bem
elevado. A Tabela abaixo mostra os preços de alguns acionadores externos
disponíveis no mercado.

\begin{table}[!h]
\centering
\caption{Acionadores externos comerciais.}
\label{tab:ibge}
\def\arraystretch{1.25}
\begin{tabular}{lccr}
	\hline
	\hline
	\textbf{Nome} & \textbf{Método de acionamento} & \textbf{Comunicação} & \textbf{Custo (USD)} \\
	\hline
	Big Candy Corni~\cite{CandyCorn}            & Aproximação  & Jack 3.5mm   & 215              \\
	Pal Pad~\cite{PalPad}                       & Pressão      & Jack 3.5mm   &  48.75 à 61.95   \\
	Jelly Bean~\cite{JellyBean}                 & Pressão      & Jack 3.5mm   &   65             \\
	Chin Switch~\cite{Chin}                     & Pressão      & Jack 3.5mm   & 220              \\
	Micro Light~\cite{MicroLight}               & Toque        & Jack 3.5mm   & 85               \\ 
	HoneyBee~\cite{HoneyBee}                    & Aproximação  & Jack 3.5mm   & 149              \\
	AbleNet string Switch~\cite{StringSwitch}   & \textcolor{red}{Puxa corda} & Jack 3.5mm & 65  \\
	Blue2 Switch~\cite{Blue2}                   & Pressão      & Bluetooth    & 185              \\
	Savant Elite2~\cite{SavantElite2}           & Pressão      & USB          & 38 à 181         \\
	Foot Pedal~\cite{FootPedal}                 & Pressão      & USB          & 267              \\
	Foot Switch~\cite{FootSwitch}               & Pressão      & USB          & 26               \\
	Sip/Puff Switch~\cite{SipPuff}              & \textcolor{red}{Sugar ou soprar} & USB & 319.8  \\  
	
	\hline
	\hline
\end{tabular}
\end{table}

\textcolor{red}{Esses que estão em vermelho eu ainda vou pensar em outras
palavras mais bonitas e trocar. Falta ainda colocar as referencias de cada um
desses acionadores.}

Acionadores que possuem como saída de comunicação o Jack 3.5mm
como~\cite{CandyCorn}, ~\cite{PalPad}, ~\cite{JellyBean}, ~\cite{Chin}, ~\cite{
MicroLight}, ~\cite{HoneyBee} e~\cite{StringSwitch} são mais utilizados como
atuadores de um determinado circuito. Um grande exemplo disso pode ser visto
no vídeo~\cite{ATswitchYT} que mostra a ativação da fala programada de uma
boneca através do pressionamento de um acionador. Como forma de controle de uma
determinada função do computador --- como o clique de um mouse --- através de um 
acionador externo, não foi encontrado nenhum dispositivo que utiliza a comunicação Jack 3.5mm conectado
diretamente no computador que realiza essa tarefa. É até possível controlar o
clique de um mouse com a comunicação Jack 3.5mm, mas é necessário o auxílio de
um mouse, como~\cite{MouseJack}, que possua uma adaptação que receba como
entrada o Jack 3.5mm de um acionador. Já para acionadores
como~\cite{Blue2}, ~\cite{SavantElite2}, ~\cite{FootPedal}, ~\cite{FootSwitch} e
~\cite{SipPuff} que possuem comunicação Bluetooth ou USB (\textit{Universal
Serial Bus}) conseguem realizar o controle dos evento de clique de mouse
facilmente sem o auxílio de outros dispositivosm, porém, acionadores que
utilizam essas comunicações são geralmente mais caros que os acionadores que
utilizam a comunicação Jack 3.5mm.

Nesse sentido, esta pesquisa tem como intuito apresentar uma solução para
diminuir a exclusão digital vivenciada pelas PCD, que muitas vezes não conseguem
utilizar aparelhos eletrônicos como \textit{smartphones} e computadores devido a 
limitação de recursos que se adaptem às suas necessidades. O uso de
acionadores externos são bons exemplos de dispositivos que auxiliam o uso de certos
aparelhos eletrônicos, porém como grande acionadores disponíveis no mercado 
possuem um custo muito elevado, há a necessidade de soluções alternativas mais
acessíveis economicamente para que mais PCD possam ter acesso a essas
ferramentas que auxiliam o uso de tarefas realizadas frequentemente em
computadores como o evento de clique simples e duplo de um mouse.  

\subsection{Trabalhos Relacionados}

\textcolor{red}{Falta adicionar as revisões dos trabalhos pesquisados até que já foi feita e ta
na pasta Revisão\_Bibliografica\_tcc. Depois é só finalizar com um parágrafo
dizendo de forma breve o que vai ser proposto. Ai finaliza a parte de
justificativa}
\section{Objetivos}

\section{Síntese de Conteúdo}


\end{chapter}

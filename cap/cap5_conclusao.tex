\begin{chapter}{Considerações Finais}

Este trabalho apresentou uma proposta de um dispositivo \textit{open-source},
cujo método de entrada é baseado em sopro, que pode ser utilizado como método
alternativo para o controle de eventos de clique de um \textit{mouse}. Para
garantir que o projeto seja de baixo custo, um \textit{software} foi desenvolvido
para comunicar o dispositivo proposto com o computador através da interface de
áudio, eliminando a possibilidade de utilizar um microcontrolador para realizar
essa tarefa. O código fonte do \textit{software} e o projeto completo do
\textit{hardware} estão disponíveis abertamente público no 
GitHub~\cite{ErickGit}.

Com base na opinião dos voluntários e no menor número de erros de cliques
realizados, pode-se concluir que o protótipo desenvolvido neste trabalho é uma
alternativa melhor que o \textit{dwell time} para executar a ação
de clique. Verificou-se que o dispositivo baseado em sopro proposto oferece mais
liberdade para os usuários realizarem tarefas como assistir a um vídeo ou
navegar em um \textit{site} de notícias, apesar das limitações impostas pelo
\textit{software} eViacam ao tentar alcançar os cantos da tela e posicionar o
cursor em itens que se alteram dinamicamente conforme o ponteiro se move.

É importante mencionar que o protótipo desenvolvido foi inspirado em diversos
acionadores externos que têm como saída o conector de áudio P2 Jack. Esses
dispositivos são utilizados em vários contextos diferentes para ser um
instrumento de interação, geralmente substituindo botões ou teclas, que muitas
vezes são uma barreira para as PCD. Portanto, o dispositivo projetado neste
trabalho pode ser utilizado como um ferramenta de propósito geral no contexto de
TA para ajudar PCD a realizarem interações de forma mais
independente com outros dispositivos. Além disso, o dispositivo proposto pode
ser uma boa alternativa aos acionadores externos comercializados, pois a maioria
desses dispositivos possuem um alto preço no mercado.

\begin{section}{Contribuições}

O desenvolvimento de acionadores externos está em andamento desde o início de 2017,
onde um acionador baseado em proximidade foi construído para ser utilizado como
uma ferramenta de ativação de um sistema que permtia o controle de um televisor
através dos movimentos da cabeça do usuário. Esse sistema foi publicado, em
formato de artigo completo, no XVI Simpósio Brasileiro sobre Fatores Humanos em
Sistemas Computacionais (IHC
2017)\footnote{\url{https://dl.acm.org/citation.cfm?id=3160504}}, sob o
título de \textit{``A Proposal  of a Universal Remote Control System Based on Head
Movements''}~\cite{Erick17}. O sistema desenvoldido não era totalmente
\textit{hands-free}, pois ainda era necessário o uso das mão para utilizar o
acionador externo baseado em aproximação.

Agora em 2018 o acionador baseado em sopro desenvolvido neste trabalho foi
projetado para ajudar as pessoas a utilizarem o clique do \textit{mouse} através
da ação de sopro, eliminado a necessidade do uso das mão. O
projeto foi aceito para apresentação, em pôster, sob o título de \textit{``A
Non-Conventional Interaction on Computational Systems
Based on Mouth Puffing''},  no XVII Simpósio Brasileiro
sobre Fatores Humanos em Sistemas Computacionais (IHC
2018)~\footnote{\url{http://www.ihc2018.ufpa.br/}}, que acontecerá entre os dias
22 e 26 de Outubro de 2018 em Belém do Pará.

\end{section}



\begin{section}{Trabalhos Futuros}

A ferramenta baseada em sopro desenvolvida neste trabalho foi inicialmente
destinada a ser usada nos sistemas operacionais Windows, Linux e Android. No
entanto, o \textit{software} responsável por captar sinais de áudio e
convertê-los em eventos de cliques foi implementado apenas para plataformas
baseadas em Linux. É necessário desenvolver um software capaz de ser executado
no Windows também, provavelmente utilizando a biblioteca de \textit{stream} de
áudio ASIO que possui suporte do PortAudio. Para sistemas Android, por outro
lado, a classe \texttt{KeyEvent} tem uma implementação nativa para capturar os
cliques do botão do fone de ouvido através da \textit{flag} de evento
\texttt{KEYCODE\_HEADSETHOOK}, fazendo com que um \textit{software} especial
para capturar dados de áudio não seja mais necessário.

Para evitar os cliques involuntários causados pelo \textit{headset} utilizado
como suporte para o dispositivo baseado em sopro, uma boa alternativa seria
utilizar uma solução semelhante a~\cite{ok}, onde um tubo de PVC fica fixo na
frente do usuário. A outra extremidade do tubo seria colocada em cima do
transdutor piezoelétrico. Dessa forma, choques acidentais com o dispositivo de
sopro seriam evitados, o que resultaria na diminuição dos casos de cliques
involuntários. Contudo, ao utilizar essa solução, seria necessário substituir o
programa que realiza o controle do cursor do \textit{mouse}. O ideal seria
utilizar um \textit{software} que realiza o controle do ponteiro através dos
movimentos dos olhos, pois o tubo de PVC ficará fixo, impossibilitando o uso de
métodos que utilizem os movimentos da cabeça do usuário como controle do cursor
do \textit{mouse}. Uma segunda alternativa seria construir uma espécie de
capacete que poderia ser construído utilizando impressoras 3D, porém essa 
solução poderia aumentar o custo da ferramenta.  

Também é necessário investigar algumas possibilidades de sensores a serem
utilizados como alternativa ao piezo. Como o funcionamento desse transdutor é
causado por estresses mecânicos, qualquer choque mecânico --- a vibração causada
pela movimentação rápida da cabeça, por exemplo --- pode ser suficiente para
ativar o evento de clique. Dispositivos como~\cite{Sip} utilizam  um microfone
de eletreto como sensor para detectar a ação de sopro, então esse sensor pode
ser uma boa solução para substituir o piezo. A desvantagem de usar microfones de
eletreto é a sensibilidade à fala, um fato que não se confirma com o
dispositivo proposto neste trabalho, que não permite a ativação do clique através
da pressão do ar produzida pela voz humana. Entretanto, alguns filtros analógicos
poderiam ser utilizados para garantir a confiabilidade do microfone de eletreto
utilizado como sensor de sopro. 

Outra possibilidade seria usar um sensor de pressão atmosférica (barômetro) para
melhorar a detecção de sopro. O sensor BMP180~\cite{bmp180} é amplamente
utilizado em projetos eletrônicos devido seu baixo custo e facilidade de uso.
Portanto, esse sensor poderia ser uma boa alternativa ao piezo.

%Seria interessante utilizar o software desenvolvido em conjunto com outros tipos
%de interação não convencional, a fim de expandir as possibilidades de interação,
%tornando o dispositivo multimodal. Acionadores  que utilizam circuitos
%eletromiográficos podem ser usados como um método alternativo de clique, pois
%há diversos trabalhos que mostram a eficácia deste tipo de método para tarefas
%que demandam um funcinamento de lógica binário. Métodos baseados em aproximação,
%como o acionador construído em~\cite{Batista17}, e até mesmo em dispositivos
%que necessitam de contato físico, como~\cite{JellyBean}, poderiam ser utilizados como
%método de interação. Dessa forma, o usuário seria capaz de escolher o
%método de interação que melhor se adapta às suas necessidades, o que acabaria 
%abrangendo um maior número de deficiências atendidas.

\end{section}

\end{chapter}

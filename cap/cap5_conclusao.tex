\begin{chapter}{Considerações Finais}
Este trabalho apresentou uma proposta de um dispositivo \textit{open-source},
cujo método de entrada é baseado em sopro, que pode ser utilizado como método
alternativo para o controle de eventos de clique de um \textit{mouse}. Para
garantir que o projeto seja d baixo custo, um \textit{software} foi desenvolvido
para comunicar o dispositivo proposto com o computador através da interface de
aúdio. O código fonte e o projeto completo do \textit{hardware} estão
disponíveis ao público no GitHub.

Com base na opinião dos voluntários e no menor numero de erros de cliques
realizados, pode-se concluir que o protótipo desenvolvido neste trabalho é uma
alternativa melhor que o método do \textit{dwell time} para a execução da ação
de clique. Verificou-se que o dispositivo baseado em sopro proposto oferece mais
liberdade para os usuários realizarem tarefas como assistir a um vídeo ou navegar
em um \textit{site} de notícias apesar das limitações impostas pelo
\textit{software} eViacam ao tentar alcaçar os cantos da tela e posicionar o
cursor no alvo correto em itens que se alteram dinâmicamente conforme o ponteiro
passa sobre esses itens, respectivamente.

É importante mencionar que o protótipo desenvolvido foi inspirado por diversos
acionadores externos que têm como sáida o conector de áudio P2 Jack. Esses
dispositivos são tilizados em vários contextos diferentes para ser um
instrumento de interação que geralmente substitui botões ou teclas, que muitas
vezes são uma barreira para as PCD. Portanto o dispositivo projetado neste
trabalho pode ser utilziado como um ferramenta de propósito geral no contexto de
Tecnologia Assistiva para ajudar as PCD realizarem interações de forma
independente com outros dispositivos. Além disso, pode ser uma boa alternativa
aos acionadores externos comerciailizados, pois a maioria desses dispositivos
possuem um alto preço no mercado.

\begin{section}{Trabalhos Futuros}

A ferramenta baseada em sopro desenvolvida neste trabalho foi inicialmente
destinada a ser usada nos sistemas operacionais Windows, Linux e Android. No
entanto, o \textit{software} responsável por captar sinais de áudio e
convertê-los em eventos de cliques foi implementado apenas para plataformas
baseadas em Linux. É necessário desenvolver um software capaz de rodar no
Windows também, provavelmente utilizando a biblioteca de \textit{stream} de
áudio ASIO~\cite{asio} que possui suporte do PortAudio~\cite{portaudio} -- uma
biblioteca \textit{open-source} que permite manipular os sinais da interface de
áudio utilizado a linguagem C++	 Para sistemas Android, por outro lado, a classe
\emph{KeyEvent} tem uma implementação nativa para capturar os cliques do botão
do fone de ouvido através da \textit{flag} de evento
\emph{KEYCODE\_HEADSETHOOK}, fazendo com que o \textit{software} que captura os
dados de áudio não seja mais necessário.

Para evitar os cliques involuntários causados pelo \textit{headset} utilizado
como suporte para o dispositivo baseado em sopro, uma boa alternativa seria
utilizar uma solução semelhante a~\cite{ok}, onde um tubo de PVC fica fixo na
frente do usuário. A outra extremidade do tubo seria colocado em cima do
transdutor piezoelétrico. Dessa forma, choques acidentais com o dispositivo de
sopro seriam evitados, o que resultaria na diminuição dos casos de cliques
involuntários. Contudo, ao utilizar essa solução será necessário substituir o
programa que realiza o controle do cursor do \textit{mouse}. O ideal seria
utilizar um \textit{software} que realiza o controle do ponteiro através dos
movimentos dos olhos, pois o tubo de PCV deverá fixo, impossibilitando o uso de
métodos que utilizem os movimentos da cabeça do usuário como controle do cursor
do \textit{mouse}. Uma segunda alternativa seria construir uma espécie de
capacete qe poderia ser construído utilizando impressoras 3D. Esse capacete
ficaria posicionado na cabeça do usuário, porém essa solução poderia causar
desconforto nos usuários.  

Também é necessário investigar algumas possibilidades de sensores a serem
utilizados como alternativa ao piezo. Como o funcionamento desse transdutor é
causado por estresses mecânicos, qualquer choque mecânico --- a vibração causada
pela movimentação rápida da cabeça, por exemplo --- pode ser suficiente para
ativar o evento de clique. Dispositivos como ~\cite{ok} utilizam  um microfone
de eletreto como sensor para detectar a ação de sopro, então esse sensor pode
ser uma boa solução para substituir o piezo. A desvantagem de usar microfones de
eletreto é que eles são sensíveis a fala, um fato que não se confirma com o
dispositivo proposto neste trabalho que não permite a ativação do clique através
da pressão do ar produzida pela voz humana. Contudo, alguns filtros analógicos
poderiam ser utilizados para garantir a confiabilidade do microfone de eletreto
utilizado como sensor de sopro. 

Outra possibilidade seria usar um sensor de pressão atmosférica (barômetro) para
melhorar a detecção de sopro. O sensor BMP180~\cite{bmp180} é amplamente
utilizado em projetos eletrônicos devido seu baixo custo e facilidade de uso.
Portanto esse sensor poderia ser uma boa alternativa ao piezo.

Seria interessante utilizar o software desenvolvido em conjunto com outros tipos
de interação não convencional, a fim de expandir as possibilidades de interação,
tornando o dispositivo multimodal. Acionadores  que utilizam circuitos
eletromiográficos podem ser usados como um método alternativo de clique, pois
há diversos trabalhos que mostram a eficácia deste tipo de método para tarefas
que demandam um funcinamento de lógica binário. Métodos baseados em aproximação
como o acionador construído em~\cite{Batista17} e até mesmo em dispositivos
que necessitam de contato físico como~\cite{ok} poderiam ser utilizados como
método de interação. Dessa forma, o usuário seria capaz de escolher o melhor
método de interação que se adpta às suas necessidades, o que acabaria ajudando
um maior número de deficiências.

\end{section}

\end{chapter}

\begin{chapter}{Considerações Finais}
Este trabalho apresentou uma proposta de um dispositivo \textit{open-source},
cujo método de entrada é baseado em sopro, que pode ser utilizado como método
alternativo para o controle de eventos de clique de um \textit{mouse}. Para
garantir que o projeto seja d baixo custo, um \textit{software} foi desenvolvido
para comunicar o dispositivo proposto com o computador através da interface de
aúdio. O código fonte e o projeto completo do \textit{hardware} estão
disponíveis ao público no GitHub.

Com base na opinião dos voluntários e no menor numero de erros de cliques
realizados, pode-se concluir que o protótipo desenvolvido neste trabalho é uma
alternativa melhor que o método do \textit{dwell time} para a execução da ação
de clique. Verificou-se que o dispositivo baseado em sopro proposto oferece mais
liberdade para os usuários realizarem tarefas como assistir a um vídeo ou navegar
em um \textit{site} de notícias apesar das limitações impostas pelo
\textit{software} eViacam ao tentar alcaçar os cantos da tela e posicionar o
cursor no alvo correto em itens que se alteram dinâmicamente conforme o ponteiro
passa sobre esses itens, respectivamente.

É importante mencionar que o protótipo desenvolvido foi inspirado por diversos
acionadores externos que têm como sáida o conector de áudio P2 Jack. Esses
dispositivos são tilizados em vários contextos diferentes para ser um
instrumento de interação que geralmente substitui botões ou teclas, que muitas
vezes são uma barreira para as PCD. Portanto o dispositivo projetado neste
trabalho pode ser utilziado como um ferramenta de propósito geral no contexto de
Tecnologia Assistiva para ajudar as PCD realizarem interações de forma
independente com outros dispositivos. Além disso, pode ser uma boa alternativa
aos acionadores externos comerciailizados, pois a maioria desses dispositivos
possuem um alto preço no mercado.

\begin{section}{Trabalhos Futuros}

\end{section}

\end{chapter}
